% Options for packages loaded elsewhere
\PassOptionsToPackage{unicode}{hyperref}
\PassOptionsToPackage{hyphens}{url}
%
\documentclass[
]{article}
\usepackage{lmodern}
\usepackage{amssymb,amsmath}
\usepackage{ifxetex,ifluatex}
\ifnum 0\ifxetex 1\fi\ifluatex 1\fi=0 % if pdftex
  \usepackage[T1]{fontenc}
  \usepackage[utf8]{inputenc}
  \usepackage{textcomp} % provide euro and other symbols
\else % if luatex or xetex
  \usepackage{unicode-math}
  \defaultfontfeatures{Scale=MatchLowercase}
  \defaultfontfeatures[\rmfamily]{Ligatures=TeX,Scale=1}
\fi
% Use upquote if available, for straight quotes in verbatim environments
\IfFileExists{upquote.sty}{\usepackage{upquote}}{}
\IfFileExists{microtype.sty}{% use microtype if available
  \usepackage[]{microtype}
  \UseMicrotypeSet[protrusion]{basicmath} % disable protrusion for tt fonts
}{}
\makeatletter
\@ifundefined{KOMAClassName}{% if non-KOMA class
  \IfFileExists{parskip.sty}{%
    \usepackage{parskip}
  }{% else
    \setlength{\parindent}{0pt}
    \setlength{\parskip}{6pt plus 2pt minus 1pt}}
}{% if KOMA class
  \KOMAoptions{parskip=half}}
\makeatother
\usepackage{xcolor}
\IfFileExists{xurl.sty}{\usepackage{xurl}}{} % add URL line breaks if available
\IfFileExists{bookmark.sty}{\usepackage{bookmark}}{\usepackage{hyperref}}
\hypersetup{
  hidelinks,
  pdfcreator={LaTeX via pandoc}}
\urlstyle{same} % disable monospaced font for URLs
\usepackage{longtable,booktabs}
% Correct order of tables after \paragraph or \subparagraph
\usepackage{etoolbox}
\makeatletter
\patchcmd\longtable{\par}{\if@noskipsec\mbox{}\fi\par}{}{}
\makeatother
% Allow footnotes in longtable head/foot
\IfFileExists{footnotehyper.sty}{\usepackage{footnotehyper}}{\usepackage{footnote}}
\makesavenoteenv{longtable}
\setlength{\emergencystretch}{3em} % prevent overfull lines
\providecommand{\tightlist}{%
  \setlength{\itemsep}{0pt}\setlength{\parskip}{0pt}}
\setcounter{secnumdepth}{-\maxdimen} % remove section numbering

\author{}
\date{}

\begin{document}

\hypertarget{template-readme-and-guidance}{%
\section{Template README and
Guidance}\label{template-readme-and-guidance}}

\begin{quote}
INSTRUCTIONS: This README suggests structure and content that would be
seen favorably by the AEA Data Editor. It may be easier to re-use if
using the
\href{https://github.com/AEADataEditor/aea-de-guidance/blob/master/template-README.md}{unformatted
version on Github}. In practice, there are many variations and
complications, and authors should feel free to adapt to their needs. All
lines starting with \texttt{\textgreater{}\ INSTRUCTIONS} can (should)
be removed from the final README. This README is in
\href{https://en.wikipedia.org/wiki/Markdown}{Markdown-format}, but can
just as easily be saved as \texttt{txt} (ASCII) file. Editing in Word or
other word processors is fine (but overkill), as long as a PDF is
submitted as well.
\end{quote}

\hypertarget{data-availability-statements}{%
\subsection{Data Availability
Statements}\label{data-availability-statements}}

\begin{quote}
INSTRUCTIONS: Every README should contain a description of the location
and accessibility of the data used in the article. These descriptions
are generally referred to as ``Data Availability Statements'' (DAS).
This should include ALL data, regardless of whether they are provided as
part of the replication archive or not, and regardless of size or scope.
For instance, if using deflators, the source of the deflators (e.g.~at
the national statistical office) should also be listed here. DAS can be
complex and varied. Examples are provided
\href{https://social-science-data-editors.github.io/guidance/Requested_information_dcas.html}{here},
and below.
\end{quote}

\begin{quote}
INSTRUCTIONS: If providing a datafile per data source, list them here;
if providing combined/derived datafiles, list them separately after the
DAS.
\end{quote}

\begin{quote}
INSTRUCTIONS: DAS do not replace Data Citations (see
\href{https://social-science-data-editors.github.io/guidance/Data_citation_guidance.html}{Guidance}).
Rather, they augment them. Depending on journal requirements and to some
extent stylistic considerations, data citations should appear in the
main article, in an appendix, or in the README. However, data citations
only provide information \textbf{where} to find the data, not
\textbf{how to access} that data. Thus, DAS augment data citations by
going into additional detail that allow a researcher to assess cost,
complexity, and availability over time of the data used by the original
author.
\end{quote}

\hypertarget{example-for-public-use-data}{%
\subsubsection{Example for public use
data}\label{example-for-public-use-data}}

\begin{quote}
The {[}DATA TYPE{]} data used to support the findings of this study have
been deposited in the {[}NAME{]} repository ({[}DOI or OTHER PERSISTENT
IDENTIFIER{]}).
{[}\href{https://www.hindawi.com/research.data/\#statement.templates}{1}{]}
\end{quote}

\hypertarget{example-for-public-use-data-with-required-registration}{%
\subsubsection{Example for public use data with required
registration:}\label{example-for-public-use-data-with-required-registration}}

\begin{quote}
The paper uses IPUMS Terra data (Ruggles et al, 2018). IPUMS-Terra does
not allow for redistribution, except for the purpose of replication
archives. Permissions as per https://terra.ipums.org/citation have been
obtained, and are documented within the ``data/IPUMS-terra'' folder.
\textgreater{} Note: the reference to ``Ruggles et al, 2018'' would be
resolved in the Reference section of this README, \textbf{and} in the
main manuscript.
\end{quote}

Datafile: \texttt{data/raw/ipums\_terra\_2018.dta}

\hypertarget{example-for-confidential-data}{%
\subsubsection{Example for confidential
data:}\label{example-for-confidential-data}}

\begin{quote}
INSTRUCTIONS: Citing and describing confidential data, in particular
when it does not have a regular distribution channel or online landing
page, can be tricky. A citation can be crafted
(\href{https://social-science-data-editors.github.io/guidance/FAQ.html\#data-citation-without-online-link}{see
guidance}), and the DAS should describe how to access, whom to contact
(including the role of the particular person, should that person
retire), and other relevant information, such as required citizenship
status or cost.
\end{quote}

\begin{quote}
The data for this project (DESE, 2019) are confidential, but may be
obtained with Data Use Agreements with the Massachusetts Department of
Elementary and Secondary Education (DESE). Researchers interested in
access to the data may contact {[}NAME{]} at {[}EMAIL{]}, also see
www.doe.mass.edu/research/contact.html. It can take some months to
negotiate data use agreements and gain access to the data. The author
will assist with any reasonable replication attempts for two years
following publication.
\end{quote}

\hypertarget{example-for-confidential-census-bureau-data}{%
\subsubsection{Example for confidential Census Bureau
data}\label{example-for-confidential-census-bureau-data}}

\begin{quote}
All the results in the paper use confidential microdata from the U.S.
Census Bureau. To gain access to the Census microdata, follow the
directions here on how to write a proposal for access to the data via a
Federal Statistical Research Data Center:
https://www.census.gov/ces/rdcresearch/howtoapply.html. You must request
the following datasets in your proposal: 1. Longitudinal Business
Database (LBD), 2002 and 2007 2. Foreign Trade Database -- Import (IMP),
2002 and 2007 {[}\ldots{]}
\end{quote}

(adapted from \href{https://doi.org/10.1093/restud/rdw057}{Fort (2016)})

\hypertarget{example-for-preliminary-code-during-the-editorial-process}{%
\subsubsection{Example for preliminary code during the editorial
process}\label{example-for-preliminary-code-during-the-editorial-process}}

\begin{quote}
Code for data cleaning and analysis is provided as part of the
replication package. It is available at
https://dropbox.com/link/to/code/XYZ123ABC for review. It will be
uploaded to the {[}JOURNAL REPOSITORY{]} once the paper has been
conditionally accepted.
\end{quote}

\hypertarget{dataset-list}{%
\subsection{Dataset list}\label{dataset-list}}

\begin{quote}
INSTRUCTIONS: In some cases, authors will provide one dataset (file) per
data source, and the code to combine them. In others, in particular when
data access might be restrictive, the replication package may only
include derived/analysis data. Every file should be described. This can
be provided as a Excel/CSV table, or in the table below.
\end{quote}

\begin{longtable}[]{@{}llll@{}}
\toprule
\begin{minipage}[b]{0.26\columnwidth}\raggedright
Data file\strut
\end{minipage} & \begin{minipage}[b]{0.19\columnwidth}\raggedright
Source\strut
\end{minipage} & \begin{minipage}[b]{0.23\columnwidth}\raggedright
Notes\strut
\end{minipage} & \begin{minipage}[b]{0.21\columnwidth}\raggedright
Provided\strut
\end{minipage}\tabularnewline
\midrule
\endhead
\begin{minipage}[t]{0.26\columnwidth}\raggedright
\texttt{data/raw/lbd.dta}\strut
\end{minipage} & \begin{minipage}[t]{0.19\columnwidth}\raggedright
LBD\strut
\end{minipage} & \begin{minipage}[t]{0.23\columnwidth}\raggedright
Confidential\strut
\end{minipage} & \begin{minipage}[t]{0.21\columnwidth}\raggedright
No\strut
\end{minipage}\tabularnewline
\begin{minipage}[t]{0.26\columnwidth}\raggedright
\texttt{data/raw/terra.dta}\strut
\end{minipage} & \begin{minipage}[t]{0.19\columnwidth}\raggedright
IPUMS Terra\strut
\end{minipage} & \begin{minipage}[t]{0.23\columnwidth}\raggedright
As per terms of use\strut
\end{minipage} & \begin{minipage}[t]{0.21\columnwidth}\raggedright
Yes\strut
\end{minipage}\tabularnewline
\begin{minipage}[t]{0.26\columnwidth}\raggedright
\texttt{data/derived/regression\_input.dta}\strut
\end{minipage} & \begin{minipage}[t]{0.19\columnwidth}\raggedright
All listed\strut
\end{minipage} & \begin{minipage}[t]{0.23\columnwidth}\raggedright
Combines multiple data sources, serves as input for Table 2, 3 and
Figure 5.\strut
\end{minipage} & \begin{minipage}[t]{0.21\columnwidth}\raggedright
Yes\strut
\end{minipage}\tabularnewline
\bottomrule
\end{longtable}

\hypertarget{computational-requirements}{%
\subsection{Computational
requirements}\label{computational-requirements}}

\begin{quote}
INSTRUCTIONS: In general, the specific computer code used to generate
the results in the article will be within the repository that also
contains this README. However, other computational requirements - shared
libraries or code packages, required software, specific computing
hardware - may be important, and is always useful, for the goal of
replication. Some example text follows.
\end{quote}

\begin{quote}
INSTRUCTIONS: We strongly suggest providing setup scripts that
install/set up the environment. Sample scripts for
\href{https://github.com/gslab-econ/template/blob/master/config/config_stata.do}{Stata},
\href{https://github.com/labordynamicsinstitute/paper-template/blob/master/programs/global-libraries.R}{R},
and \href{https://pip.readthedocs.io/en/1.1/requirements.html}{Python}
are easy to set up and implement.
\end{quote}

\hypertarget{software-requirements}{%
\subsubsection{Software Requirements}\label{software-requirements}}

\begin{itemize}
\tightlist
\item
  Stata (code was last run with version 15)

  \begin{itemize}
  \tightlist
  \item
    \texttt{estout} (as of 2018-05-12)
  \item
    \texttt{rdrobust} (as of 2019-01-05)
  \item
    the program ``\texttt{0\_setup.do}'' will install all dependencies
    locally, and should be run once.
  \end{itemize}
\item
  Python 3.6.4

  \begin{itemize}
  \tightlist
  \item
    \texttt{pandas} 0.24.2
  \item
    \texttt{numpy} 1.16.4
  \item
    the file ``\texttt{requirements.txt}'' lists these dependencies,
    please run ``\texttt{pip\ install\ -r\ requirements.txt}'' as the
    first step. See
    \url{https://pip.readthedocs.io/en/1.1/requirements.html} for
    further instructions on using the ``\texttt{requirements.txt}''
    file.
  \end{itemize}
\item
  Intel Fortran Compiler version 20200104
\item
  Matlab (code was run with Matlab Release 2018a)
\item
  R 3.4.3

  \begin{itemize}
  \tightlist
  \item
    \texttt{tidyr} (0.8.3)
  \item
    \texttt{rdrobust} (0.99.4)
  \item
    the file ``\texttt{0\_setup.R}'' will install all dependencies
    (latest version), and should be run once prior to running other
    programs.
  \end{itemize}
\end{itemize}

Portions of the code use bash scripting, which may require Linux.

Portions of the code use Powershell scripting, which may require Windows
10 or higher.

\hypertarget{description-of-programs}{%
\subsubsection{Description of programs}\label{description-of-programs}}

\begin{quote}
INSTRUCTIONS: Give a high-level overview of the program files and their
purpose. Remove redundant/ obsolete files from the Replication archive.
\end{quote}

\begin{itemize}
\tightlist
\item
  Programs in \texttt{programs/01\_dataprep} will extract and reformat
  all datasets referenced above. The file
  \texttt{programs/01\_dataprep/master.do} will run them all.
\item
  Programs in \texttt{programs/02\_analysis} generate all tables and
  figures in the main body of the article. The program
  \texttt{programs/02\_analysis/master.do} will run them all. Each
  program called from \texttt{master.do} identifies the table or figure
  it creates (e.g., \texttt{05\_table5.do}). Output files are called
  appropriate names (\texttt{table5.tex}, \texttt{figure12.png}) and
  should be easy to correlate with the manuscript.
\item
  Programs in \texttt{programs/03\_appendix} will generate all tables
  and figures in the online appendix. The program
  \texttt{programs/03\_appendix/master-appendix.do} will run them all.
\item
  Ado files have been stored in \texttt{programs/ado} and the
  \texttt{master.do} files set the ADO directories appropriately.
\item
  The program \texttt{programs/00\_setup.do} will populate the
  \texttt{programs/ado} directory with updated ado packages, but for
  purposes of exact reproduction, this is not needed. The file
  \texttt{programs/00\_setup.log} identifies the versions as they were
  last updated.
\item
  The program \texttt{programs/config.do} contains parameters used by
  all programs, including a random seed. Note that the random seed is
  set once for each of the two sequences (in \texttt{02\_analysis} and
  \texttt{03\_appendix}). If running in any order other than the one
  outlined below, your results may differ.
\end{itemize}

\hypertarget{memory-and-runtime-requirements}{%
\subsubsection{Memory and Runtime
Requirements}\label{memory-and-runtime-requirements}}

\begin{quote}
INSTRUCTIONS: Memory and compute-time requirements may also be relevant
or even critical. Some example text follows.
\end{quote}

The code was last run on a \textbf{4-core Intel-based laptop with MacOS
version 10.14.4}.

Portions of the code were last run on a \textbf{32-core Intel server
with 1024 GB of RAM, 12 TB of fast local storage}. Computation took 734
hours.

Portions of the code were last run on a \textbf{12-node AWS R3 cluster,
consuming 20,000 core-hours}.

\hypertarget{instructions}{%
\subsection{Instructions}\label{instructions}}

\begin{quote}
INSTRUCTIONS: The first two sections ensure that the data and software
necessary to conduct the replication have been collected. This section
then describes a human-readable instruction to conduct the replication.
This may be simple, or may involve many complicated steps. It should be
a simple list, no excess prose. Examples follow.
\end{quote}

\begin{enumerate}
\def\labelenumi{\arabic{enumi}.}
\tightlist
\item
  Run \texttt{programs/00\_setup.do}, which will create all output
  directories.

  \begin{itemize}
  \tightlist
  \item
    If wishing to update the ado packages used by this archive, change
    the parameter \texttt{update\_ado} to \texttt{yes}. However, this is
    not needed to successfully reproduce the manuscript tables.
  \end{itemize}
\item
  Download the data files referenced above. Each should be stored in the
  prepared subdirectories of \texttt{data/}, in the format that you
  download them in. Do not unzip. Scripts are provided in each directory
  to download the public-use files.

  \begin{itemize}
  \tightlist
  \item
    Confidential data files requested as part of your FSRDC project will
    appear in the \texttt{/data} folder. No further action is needed on
    the replicator's part.
  \end{itemize}
\item
  Run the programs in \texttt{programs/01\_dataprep}.

  \begin{itemize}
  \tightlist
  \item
    These programs were last run at various times in 2018.
  \item
    Order does not matter, all programs can be run in parallel, if
    needed.
  \item
    A \texttt{programs/01\_dataprep/master.do} will run them all in
    sequence, which should take about 2 hours.
  \end{itemize}
\item
  Run the program \texttt{programs/02\_analysis/master.do}.

  \begin{itemize}
  \tightlist
  \item
    If running programs individually, note that ORDER IS IMPORTANT.
  \item
    The programs were last run top to bottom on July 4, 2019.
  \end{itemize}
\item
  Run the program \texttt{programs/03\_appendix/master-appendix.do}. The
  programs were last run top to bottom on July 4, 2019.
\end{enumerate}

\hypertarget{list-of-tables-and-programs}{%
\subsection{List of tables and
programs}\label{list-of-tables-and-programs}}

\begin{quote}
INSTRUCTIONS: Your programs should clearly identify the tables and
figures as they appear in the manuscript, by number. Sometimes, this may
be obvious, e.g.~a program called ``\texttt{table1.do}'' generates a
file called \texttt{table1.png}. Sometimes, mnemonics are used, and a
mapping is necessary. In all circumstances, provide a list of tables and
figures, identifying the program (and possibly the line number) where a
figure is created.
\end{quote}

\begin{longtable}[]{@{}lllll@{}}
\toprule
\begin{minipage}[b]{0.13\columnwidth}\raggedright
Figure/Table \#\strut
\end{minipage} & \begin{minipage}[b]{0.18\columnwidth}\raggedright
Program\strut
\end{minipage} & \begin{minipage}[b]{0.09\columnwidth}\raggedright
Line Number\strut
\end{minipage} & \begin{minipage}[b]{0.23\columnwidth}\raggedright
Output file\strut
\end{minipage} & \begin{minipage}[b]{0.23\columnwidth}\raggedright
Note\strut
\end{minipage}\tabularnewline
\midrule
\endhead
\begin{minipage}[t]{0.13\columnwidth}\raggedright
Table 1\strut
\end{minipage} & \begin{minipage}[t]{0.18\columnwidth}\raggedright
02\_analysis/table1.do\strut
\end{minipage} & \begin{minipage}[t]{0.09\columnwidth}\raggedright
\strut
\end{minipage} & \begin{minipage}[t]{0.23\columnwidth}\raggedright
summarystats.csv\strut
\end{minipage} & \begin{minipage}[t]{0.23\columnwidth}\raggedright
\strut
\end{minipage}\tabularnewline
\begin{minipage}[t]{0.13\columnwidth}\raggedright
Table 2\strut
\end{minipage} & \begin{minipage}[t]{0.18\columnwidth}\raggedright
02\_analysis/table2and3.do\strut
\end{minipage} & \begin{minipage}[t]{0.09\columnwidth}\raggedright
15\strut
\end{minipage} & \begin{minipage}[t]{0.23\columnwidth}\raggedright
table2.csv\strut
\end{minipage} & \begin{minipage}[t]{0.23\columnwidth}\raggedright
\strut
\end{minipage}\tabularnewline
\begin{minipage}[t]{0.13\columnwidth}\raggedright
Table 3\strut
\end{minipage} & \begin{minipage}[t]{0.18\columnwidth}\raggedright
02\_analysis/table2and3.do\strut
\end{minipage} & \begin{minipage}[t]{0.09\columnwidth}\raggedright
145\strut
\end{minipage} & \begin{minipage}[t]{0.23\columnwidth}\raggedright
table3.csv\strut
\end{minipage} & \begin{minipage}[t]{0.23\columnwidth}\raggedright
\strut
\end{minipage}\tabularnewline
\begin{minipage}[t]{0.13\columnwidth}\raggedright
Figure 1\strut
\end{minipage} & \begin{minipage}[t]{0.18\columnwidth}\raggedright
n.a. (no data)\strut
\end{minipage} & \begin{minipage}[t]{0.09\columnwidth}\raggedright
\strut
\end{minipage} & \begin{minipage}[t]{0.23\columnwidth}\raggedright
\strut
\end{minipage} & \begin{minipage}[t]{0.23\columnwidth}\raggedright
Source: Herodus (2011)\strut
\end{minipage}\tabularnewline
\begin{minipage}[t]{0.13\columnwidth}\raggedright
Figure 2\strut
\end{minipage} & \begin{minipage}[t]{0.18\columnwidth}\raggedright
02\_analysis/fig2.do\strut
\end{minipage} & \begin{minipage}[t]{0.09\columnwidth}\raggedright
\strut
\end{minipage} & \begin{minipage}[t]{0.23\columnwidth}\raggedright
figure2.png\strut
\end{minipage} & \begin{minipage}[t]{0.23\columnwidth}\raggedright
\strut
\end{minipage}\tabularnewline
\begin{minipage}[t]{0.13\columnwidth}\raggedright
Figure 3\strut
\end{minipage} & \begin{minipage}[t]{0.18\columnwidth}\raggedright
02\_analysis/fig3.do\strut
\end{minipage} & \begin{minipage}[t]{0.09\columnwidth}\raggedright
\strut
\end{minipage} & \begin{minipage}[t]{0.23\columnwidth}\raggedright
figure-robustness.png\strut
\end{minipage} & \begin{minipage}[t]{0.23\columnwidth}\raggedright
Requires confidential data\strut
\end{minipage}\tabularnewline
\bottomrule
\end{longtable}

\hypertarget{references}{%
\subsection{References}\label{references}}

\begin{quote}
INSTRUCTIONS: As in any scientific manuscript, you should have proper
references. For instance, in this sample README, we cited ``Ruggles et
al, 2019'' and ``DESE, 2019'' in a Data Availability Statement. The
reference should thus be listed here, in the style of your journal:
\end{quote}

Steven Ruggles, Steven M. Manson, Tracy A. Kugler, David A. Haynes II,
David C. Van Riper, and Maryia Bakhtsiyarava. 2018. ``IPUMS Terra:
Integrated Data on Population and Environment: Version 2
{[}dataset{]}.'' Minneapolis, MN: \emph{Minnesota Population Center,
IPUMS}. https://doi.org/10.18128/D090.V2

Department of Elementary and Secondary Education (DESE), 2019. ``Student
outcomes database {[}dataset{]}'' \emph{Massachusetts Department of
Elementary and Secondary Education (DESE)}. Accessed January 15, 2019.

\hypertarget{acknowledgements}{%
\section{Acknowledgements}\label{acknowledgements}}

Some content on this page was copied from
\href{https://www.hindawi.com/research.data/\#statement.templates}{Hindawi}.
Other content was adapted from
\href{https://doi.org/10.1093/restud/rdw057}{Fort (2016)}, Supplementary
data, with the author's permission.

\end{document}
